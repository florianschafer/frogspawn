%----------------------------------------------------------------------------------------
%	PACKAGES AND OTHER DOCUMENT CONFIGURATIONS
%----------------------------------------------------------------------------------------

\documentclass[10pt, a4paper, twocolumn]{article} % 10pt font size (11 and 12 also possible), A4 paper (letterpaper for US letter) and two column layout (remove for one column)

\input{structure.tex} % Specifies the document structure and loads requires packages

%----------------------------------------------------------------------------------------
%	ARTICLE INFORMATION
%----------------------------------------------------------------------------------------

\title{Nephilia: A new approach to recursive spectral clustering of bipartite graphs} % The article title

\author{
	\authorstyle{Florian Schaefer}
}

% Example of a one line author/institution relationship
%\author{\newauthor{John Marston} \newinstitution{Universidad Nacional Autónoma de México, Mexico City, Mexico}}

\date{\today} % Add a date here if you would like one to appear underneath the title block, use \today for the current date, leave empty for no date

%----------------------------------------------------------------------------------------

\begin{document}

\maketitle % Print the title

\thispagestyle{firstpage} % Apply the page style for the first page (no headers and footers)

%----------------------------------------------------------------------------------------
%	ABSTRACT
%----------------------------------------------------------------------------------------

\lettrineabstract{There goes the abstract. Please fill it with some live!}

%----------------------------------------------------------------------------------------
%	ARTICLE CONTENTS
%----------------------------------------------------------------------------------------

\section{Spectral shifting}
\paragraph
As pointed out earlier, we're only interested in the second-smallest eigenvalue of the normalized laplacian.
While there exist many libraries (ARPack, Eigen, SLEPc, just to name a few) to perform this task, due to their
generality, they lack the ability to exploit certain properties of the problem at hand. More precisely, both the extremal
eigenvalues as well as the eigenvector associated with the smallest eigenvalue are known a priori.
\\
It shows that those properties can easily exploited to transform the matrix at hand such that the new matrix
associates the desired eigenvector with its largest eigenvalue. This is performed using a well-known technique called spectral shifting.

% TODO: Check matrix conditions for both theorems (symmetric? positive-definite? .... v != 0, ...)
\subsection{Spectral transformations}

% https://math.stackexchange.com/questions/2214641/shifting-eigenvalues-of-a-matrix
\newtheorem{Brauer}[]{Theorem (Brauer)}[section]
\begin{Brauer}
	Let $A \in \mathbb{R}^{n \times n}$ a matrix with $Av=\lambda v$ for some eigenvalue $\lambda \in \mathbb{R}$ and $0 \neq v \in \mathbb{R}^n$.
	Furthermore, let $r \in \mathbb{R}^n$ with $r^{T}v=1$. Then, $\forall \mu \in \mathbb{R}:$ the eigenvalues of
	\begin{align}
		\hat{A} := A + (\lambda - \mu)vr^T
	\end{align}
	are exactly those of $A$, except for $\lambda$, which has been replaced by $\mu$. Furthermore, the eigenvector $v$ remains
	completely unchanged, i.e. $\hat{A}v = \mu v$
\end{Brauer}
This theorem will allow us later to swap specific eigenvalues.

\newtheorem{Brauer Param Note}[]{Note}[section]
\begin{Brauer Param Note}
Chosing $r:=v$ in the above theorem is explicitly allowed if $v$ is normalized.
\end{Brauer Param Note}

\newtheorem{Spectral Shifting}[]{Theorem (Spectral Shifting)}[section]
\begin{Spectral Shifting}
	Let $A \in \mathbb{R}^{n \times n}$ a symmetric, positive-definite matrix with $Av=\lambda v$ for some eigenvalue $\lambda \in \mathbb{R}$.
	Furthermore, let $\mu \in \mathbb{R}$. Then
	\begin{align}
		(A - \sigma I)v = (\lambda - \mu)v
	\end{align}
That is, $(\lambda - \mu)v$ is an eigenvalue of the Matrix $(A - \sigma I)v$. Just like before, this does not affect the
eigenvector $v$.
\end{Spectral Shifting}

\newtheorem{Reversal}[]{Corollary (Reverse Spectra)}[section]
\begin{Reversal}
	Obviously the reversed statement $(\sigma I - A)v = (\mu - \lambda)v$ also holds true.
	Let $A \in \mathbb{R}^{n \times n}$, $v$ like above. Let $\lambda_n$ the largest eigenvalue.
	It immediately follows that the transformation
	\begin{align}
		\hat{A} := \lambda_n I - A
	\end{align}
	will mirror the eigenvalue at $\lambda_n$. That is, the order of the spectrum is reversed without affecting any eigenvectors of $A$.
\end{Reversal}
With those tools at hand, we can now formulate the transformation that will assign the eigenvector of the
second-smallest eigenvalue to the largest eigenvalue of the transformed matrix.
\\
Remember that there a few known properties of the normalized laplacian spectrum. In particular:
\begin{itemize}
	\item We know that the smallest eigenvalue is $0$ with eigenvector $\frac{D^\frac{1}{2}}{\lVert D^\frac{1}{2} \rVert}$ % TODO: Wrong! There is no D here
	\item If the graph is bipartite, the largest eigenvalue is known to be $2$.
\end{itemize}
\\
Now we can first reverse the spectrum along the largest eigenvalue $2$ and then (the order matters) send the
now-largest eigenvalue $2$ (that used to be $0$) to $0$, so that this eigenvalue ends up with multiplicity 2.
But most importantly, the largest eigenvalue and its eigenvector of this new matrix are just what we were seeking for.
\\
Asuming that $\lVert v_0 \lVert = 1$
\begin{equation}
		\hat{\mathcal{L}} := \underbrace{2I - \mathcal{L}}_{\text{reverse spectrum}} - \underbrace{2 v_0 v_0^T}_{\text{eigenvalue $2 \rightarrow 0$}}
\end{equation}


\section{The case against other eigensolvers}

%----------------------------------------------------------------------------------------
%	BIBLIOGRAPHY
%----------------------------------------------------------------------------------------

\printbibliography[title={Bibliography}] % Print the bibliography, section title in curly brackets

%----------------------------------------------------------------------------------------

\end{document}
